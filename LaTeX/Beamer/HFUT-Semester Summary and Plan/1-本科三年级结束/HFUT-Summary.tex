\documentclass[aspectratio=169]{beamer}
% 默认使用hfut-sx主题,可以自由更换
\usetheme{hfut-CambridgeUS}

\usepackage{amsmath,amsfonts,amssymb}
\usepackage{multirow}
\usepackage{booktabs}
\usepackage{layout}
\usepackage{minted}
\usepackage{calligra} % 加载 calligra 字体
\usepackage[natbib=true]{biblatex}

% biblatex 数据库
\addbibresource{ref.bib}

% 幻灯片分页时从第二页开始打印“(续)”
\setbeamertemplate{frametitle continuation}[from second]

% 每一节开始的时候打印目录
\AtBeginSection[]
{
  \begin{frame}{目录}
    \tableofcontents[currentsection]
  \end{frame}
}



\newcommand{\BibTeX}{\textsc{Bib}\TeX{}}
\newcommand{\BibLaTeX}{\textsc{Bib}\LaTeX{}}
\newcommand{\Beamer}{\textsc{Beamer}}
\newcommand{\enableindent}{\setlength{\parskip}{6pt}\setlength{\parindent}{2em}}
% algorithm
\usepackage{algorithm}
\usepackage{algorithmicx,algpseudocode}
\floatname{algorithm}{算法}

\newcommand{\contactinfo}{
  \begin{center}
    Email: \href{mailto:jieyu8258@gmail.com}{jieyu8258@gmail.comemail}
  \end{center}
}


\title{本学期总结与下学期规划}
% \subtitle{副标题}
\author{王子颉}
\institute{School of Computer and Informatics, HFUT}
\date{\today}

\begin{document}

\begin{frame}
	\maketitle
	\contactinfo
\end{frame}

\begin{frame}{目录}
	\tableofcontents
\end{frame}

\section{本学期总结}

\begin{frame}
  \begin{itemize}
    \item 完成了本学期的专业课学习,成功拿到保研名额并进入了胡老师的课题组。\pause
    \item 听从胡老师的建议,看完了李沐的《动手学习深度学习》课程,动手敲了里面的一些代码,对深度学习里的一些基础网络架构有了一定了解。\pause
    \item 初步的完成了毕业论文撰写,后续更多的时间跟进课题组的工作。
  \end{itemize}
\end{frame}

\section{下学期规划}

\begin{frame}
  \begin{itemize}
    \item 继续深入学习深度学习相关的知识,包括但不限于网络架构、优化算法、损失函数等。\pause
    \item 跟进课题组的工作,完成课题组的任务。\pause
    \item 继续完善毕业论文,争取在下学期初完成论文的撰写。
  \end{itemize}
\end{frame}

\section{结束}
% \begin{frame}[allowframebreaks]
% 	\frametitle{参考文献}
% 	{
% 		\tiny
% 		\nocite{*}
% 		\printbibliography[heading=none]
% 	}
% \end{frame}

\begin{frame}
	\begin{center}
    {\Huge\calligra Thanks for your attention!}
  \end{center}
\end{frame}

\end{document}
