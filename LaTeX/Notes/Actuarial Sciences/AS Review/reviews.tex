\documentclass[lang=cn,10pt]{elegantbook}
\usepackage{subcaption}


\newcommand\bv[1]{\boldsymbol{#1}}
\newcommand\mb[1]{\mathbb{#1}}
\newcommand\mc[1]{\mathcal{#1}}

\title{Actuarial Sciences Review}
% \subtitle{Elegant\LaTeX{} 经典之作}

\author{Lollins}
\institute{安徽师范大学数学与统计学院}
\date{\today}
% \version{4.3}
\bioinfo{邮箱}{jieyu8258@gmail.com}


\extrainfo{唯有热爱可抵岁月长!
}

\setcounter{tocdepth}{3}

\cover{star.jpg}

\logo{logo.jpg}

% 本文档命令
\usepackage{array}
\newcommand{\ccr}[1]{\makecell{{\color{#1}\rule{1cm}{1cm}}}}
% 精算学下角标符号
\DeclareRobustCommand{\annu}[1]{_{%
    \def\arraystretch{0}%
    \setlength\arraycolsep{1pt}% adjust these
    \setlength\arrayrulewidth{.2pt}% two settings
    \begin{array}[b]{@{}c|}\hline
        \\[\arraycolsep]%
        \scriptstyle #1%
    \end{array}%
}}


% 修改标题页的橙色带
\definecolor{customcolor}{RGB}{135, 206, 235}
\colorlet{coverlinecolor}{customcolor}

\begin{document}

\maketitle

\begin{center}
	\Huge\textbf{前言}
\end{center}
\par
由于king哥已经弄好了一份精美的精算学pdf电子笔记,若我再做精算学上课笔记,实乃多余。故我决定做一份精算学复习资料,主要是对精算学的一些重要知识点进行总结\footnote{本文不涉及一些细节上证明以及推导,想了解公式由来的读者直接阅读king哥群里的pdf即可。},以便于大家复习。

随着这学期的期末考试结束,大家的大学生涯基本也算是告一段落了。愿大家都能顺利通过考试,也祝大家顺利毕业,前程似锦!

\begin{flushright}
	\begin{tabular}{c}
		Lollins \\
		\today
	\end{tabular}
\end{flushright}

\newpage
\pagenumbering{Roman}
\setcounter{page}{1}
\tableofcontents
\newpage
\setcounter{page}{1}
\pagenumbering{arabic}

\chapter*{期末复习相关内容}
\section*{精算学的主要内容}
\begin{enumerate}
    \item 你能活多久?(生存分布)
    \item 你死的时候, 保险公司支付你1元, 这1元的现值为多少?(人寿保险)
    \item 在你活着时, 保险公司每年支付你1元, 这些支付的现值是多少?(生存年金)
    \item 上述的寿险与生存年金, 你该向保险公司缴纳多少保费?(保费理论)
    \item 保险公司为了保证支付, 要准备多少钱?(准备金理论)
\end{enumerate}

\section*{题型}
\begin{enumerate}
	\item 填空题: $5\times 3 = 15$分; (年金与寿险的关系, UDD假设关系)
	\item 分析题: $3\times 7 = 21$分; (解释含义)
	\item 解答题: $3\times 4 = 12$分; (生存年金分类, 第四章与第五章的一些定义)
	\item 计算题: $4\times 10 = 40$分; (前四章一章一题)
	\item 综合题: $1\times 12 = 12$分. (重点复习第五章的例题)
\end{enumerate}
\begin{remark}
	连续分布就考指数, 离散分布就考均匀, 生死合险不考, 每年分成$m$个区间不考. 
\end{remark}

\chapter{生存分布}

\section{新生儿的生存分布}

\begin{definition}[生存函数]
  称$s(t):=P(X>t),t>0$ 为$X$的生存函数.
\end{definition}

\begin{corollary}
  $s(t)=1-F(t)$, $s'(t)=-f(t)$.
\end{corollary}

\begin{definition}[死亡力函数]
  称$\mu(t):=-\frac{s'(t)}{s(t)},t\ge 0$ 为新生儿的死亡力函数.
\end{definition}

\begin{corollary} 关于$\mu(t),$ $s(t)$ 及 $f_X(t)$ 有如下结论:
\begin{enumerate}
	\item $\mu(t)=-\frac{s'(t)}{s(t)}=\frac{f_X(t)}{1-F_X(t)}=\frac{f_X(t)}{s(t)};$
	\item $f_X(t)=\mu(t)s(t);$
	\item $s(t)=e^{-\int_0^t\mu(s)ds}.$
\end{enumerate}
\end{corollary}

\begin{remark}
    \begin{enumerate}
        \item 由$s(t) = e^{-\int_{0}^{t}\mu(s)\mathrm{d}s}$ 及$ \mu(t) = -\frac{s'(t)}{s(t)}$可知, 生存函数$s(t)$与死亡力函数$\mu(t)$相互唯一确定.
        \item 一个函数$\mu(t)$要作为死亡力, 必须满足以下两条:
            \begin{enumerate}
                \item $\mu(t) \geq 0, ~\forall t \geq 0$ (保证$s(t)$单调递减).
                \item $\int_0^{\infty}\mu(t)\mathrm{d}t = \infty$(保证$s(\infty)=0$).
            \end{enumerate}
    \end{enumerate}
\end{remark}

\begin{example}
    假设新生儿的寿命服从以$\lambda$为参数的指数分布, 则密度函数$f_X(t)=\lambda e^{-\lambda t},~t>0.$
\end{example}
\begin{solution}
	分布函数$F_X(t)=\int_0^t f_X(s)ds=1-e^{-\lambda t}, t>0.$ 
	
	生存函数$s(t)=1-F_X(t)=e^{-\lambda t},t>0.$  故其死亡力函数为
    \begin{align}\label{ep}
        \mu(t)=-\frac{s'(t)}{s(t)}=-\frac{-\lambda e^{-\lambda t}}{e^{\lambda t}}\equiv\lambda.
    \end{align}
\end{solution}

\begin{remark}
    由\eqref{ep}式可知, 若新生儿寿命服从以$\lambda$ 为参数的指数分布, 则死亡力$\mu(t)\equiv \lambda,$ 和$t$ 无关. 这表示新生儿的死亡力在任何时候都是一样的. 也就是说, 新生儿永远年轻. 这当然与实际情况不符. 所以, 指数分布作为寿命分布是有缺陷的. 但由于指数分布的计算较为简单, 所以在理论研究中, 学者们很多时候都采用指数分布作为寿命分布.
\end{remark}

\begin{definition}[整数年龄与分数年龄]
	很多时候, 保险金都是在整数时刻支付的. 所以有必要研究整数年龄和分数年龄. 设$K(0)$为$X$的整数部分, $S(0)$为$X$ 的分数部分. 即
$$X = K(0) + S(0).$$
记$\mathring{e}_0 = E(X),$ 它表示新生儿的期望寿命; 记$e_0 = E(K(0)),$ 它表示期望整数寿命. 易知
\begin{equation*}
    e_0 \le \mathring{e}_0 < e_0 + 1.
\end{equation*}
\end{definition}

\begin{lemma}\label{lemm0}
    设随机变量$X$的$n$阶矩存在, 即$E(X^n) < \infty,$ 则$\lim_{M \rightarrow \infty}M^ns(M) = 0.$
\end{lemma}

\begin{corollary}如下结论成立:
    \begin{enumerate}
         \item $\mathring{e}_0 = E(X) = \int_0^{\infty}s(t)\mathrm{d}t;$
        \item $E(X^2) = \int_{0}^{\infty} 2ts(t)\mathrm{d}t;$
        \item $E(K(0)^2) = \sum_{n = 1}^{\infty} (2n-1)s(n);$
        \item $e_0=E(K(0)) = \sum_{n = 1}^{\infty} s(n).$
    \end{enumerate}
\end{corollary}

\begin{remark}
	$E(X^n)=\int_0^\infty nt^{n-1}s(t)\mathrm{d}t$.
\end{remark}

\section{$x$岁个体的生存分布}

\begin{definition}[$x$岁个体余命的分布、密度及生存函数]
	将一个$x$岁还活着的个体记为$(x).$ 个体$(x)$的余命记为$T(x)$, 显然有$T(x)=X-x$.

	记$F_{T(x)}(t)$为$T(x)$的分布函数, $f_{T(x)}(t)$为$T(x)$的密度函数, 则$$F_{T(x)}(t) = 1 - \frac{s(x+t)}{s(x)}, ~f_{T(x)}(t) = -\frac{s'(x+t)}{s(x)}.$$

	称$s_{T(x)}(t): = P(T(x)>t)$为个体 $(x)$的的生存函数.
\end{definition}

\begin{definition}[$x$岁个体的死亡力]
    称$\mu_x(t) = -\frac{s_{T(x)}'(t)}{s_{T(x)}(t)}$为$x$岁个体在$t$ 年后的死亡力函数.
\end{definition}

\begin{corollary}
	我们有
	\begin{enumerate}
		\item $s_{T(x)}(t)=1-F_{T(x)}(t)=\frac{s(x+t)}{s(x)};$
		\item $\mu_{x}(t)=\frac{f_{T(x)}(t)}{1-F_{T(x)}(t)}=\frac{f_{T(x)}(t)}{s_{T(x)}(t)}=-\frac{s_{T(x)}'(t)}{s_{T(x)}(t)}.$
		\item $f_{T(x)}(t)=s_{T(x)}(t)\mu_{x}(t);$
		\item $\mu_x(t)=\mu(x+t);$
		\item $s_{T(x)}(t)=e^{-\int_0^t \mu_x(s)ds}=e^{-\int_0^t \mu(x+s)ds}=e^{-\int_x^{x+t} \mu(s)ds}.$
	\end{enumerate}
\end{corollary}

\begin{remark}
    理论上, 一个人一旦出生, 其死亡力就``注定"了. 如果他在$x$岁还活着, 在$t$ 年后他变为$x+t$ 岁, 此时他的死亡力是$\mu_x(t).$ 换一种观点, 如果站在0 时刻(他出生时)看, 他在$x+t$岁的死亡力应为$\mu(x+t).$ 故有
    $$\mu_x(t)=\mu(x+t).$$
\end{remark}

\begin{example}
    设新生儿的寿命服从以$\lambda>0$为参数的指数分布. 则
    $s(t)=e^{-\lambda t}, t>0.$
\end{example}
\begin{solution}
	从而有
    \begin{align*}
         & F_{T(x)}(t)=1-\frac{s(x+t)}{s(x)}=1-\frac{e^{-\lambda(x+t)}}{e^{-\lambda x}}=1-e^{-\lambda t} =F_X(t); \\
         & f_{T(x)}(t)=F_{T(x)}'(t)=F_X'(t)=f_X(t);   \\
         & \mu_x(t)=\mu(x+t)\equiv\lambda.
    \end{align*}
    以上计算再次表明, 在指数分布寿命假设下, 新生儿的的寿命$X$与$x$岁的个体的余命$T(x)$的分布相同. 进一步说明指数分布作为寿命分布是有缺陷的.
\end{solution}

\begin{proposition}$\forall u,t>0,$ 有
    \begin{align}
         & P(T(x)>t+u|T(x)>t)=P(T(x+t)>u).\label{tu}
    \end{align}
    该式的含义为: 一个$x$岁的人, 在$x+t$岁还活着的条件下, 再活$u$年不死的概率与一个$x+t$ 岁的人在$u$年内未死的概率相等.
\end{proposition}

\begin{remark}由\eqref{tu}式立即可得
    \begin{align}\label{tul}
         & P(T(x)\leq t+u|T(x)>t)=P(T(x+t)\leq u).
    \end{align}
\end{remark}

\begin{example}
    设新生儿的寿命服从指数分布, 参数为$\lambda$.
\end{example}
\begin{solution}
	我们有$\mu(t)\equiv\lambda,$ 且
    \begin{align*}
         & s(t)=e^{-\lambda t},t>0.                                                                                       \\
         & F_{T(x)}=1-\frac{s(x+t)}{s(x)}=1-e^{- \lambda t}=F_x(t).   \\
         & f_{T(x)}(t)={F'}_{T(x)}(t)=\lambda e^{- \lambda t}=f_x(t),t>0.  \\
         & s_{T(x)}(t)=\frac{s(x+t)}{s(x)}=e^{-\lambda t}=s(t),t>0. \\
         & \mu_x(t)=\mu(x+t)\equiv\mu,t>0. \\
         & e_x = \sum_{k=1}^{\infty} {_kp_x}=\sum_{k=1}^{\infty} {e^{- \lambda k}}=\frac{e^{- \lambda}}{1-e^{- \lambda}}. \\
         & \mathring{e}_x=\int_{0}^{\infty}{_tp_x}dt=\int_{0}^{\infty}{e^{- \lambda t}}dt=\frac{1}{\lambda}.
    \end{align*}
    显而易见, 这里的$\mathring e_x $和$e_x$与$x$无关, 也就是说,  所有人的剩余寿命的期望都是一样的, 和他现在的年龄无关. 这进一步说明指数分布作为寿命分布是有缺陷的.   此外, 因$\mathring{e}_x=ET(x)=\frac{1}{\lambda},$ 故指数分布的参数$\lambda$正好是期望寿命的倒数.
\end{solution}

\begin{proposition}
	定义如下几个记号:
\begin{enumerate}
    \item 用$_tp_{x}\stackrel{\text{def}}{=}P(T(x)>t)=s_{T(x)}(t)$
        表示个体$(x)$在$t$年后还活着的概率. 显然有
        $$ {}_tp_x=s_{T(x)}(t)=\mathrm{e}^{-\int_{0}^{t}\mu_x(s)\mathrm{d}s}=\mathrm{e}^{-\int_{0}^{t}\mu(x+s)\mathrm{d}s}=\mathrm{e}^{-\int_{x}^{x+t}\mu(s)\mathrm{d}s}.$$
    \item 用$_tq_{x}\stackrel{\text{def}}{=}P(T(x)\leq t)=F_{T(x)}(t)$
        表示一个$x$岁的人在$t$年内死亡的概率. 易知
        $$_tp_{x}+{}_tq_{x}=1.$$
    \item 用$ _{u|t}q_x\stackrel{\text{def}}{=}P(u<T(x)\leq u+t)$
        表示一个$x$岁的人在$x+u$岁还活着, 但在未来$t$年内死亡的概率.
\end{enumerate}
\end{proposition}

\begin{corollary}
如下几个结论成立:
\begin{enumerate}
	\item $\frac{d(_tp_x)}{dt}=-_tp_x\mu_x(t);$
	\item $\frac{d(_tp_x)}{dx}= {}_tp_x(\mu(x)-\mu(x+t));$
	\item $f_{T(x)}(t)={}_tp_x\cdot \mu_x(t);$
    \item $_tp_x={}_sp_x\cdot{}_{t-s}p_{x+s},0\leq s\leq t;$
	\item ${}_{u|t}q_x={}_up_x-{}_{u+t}p_x, u,t>0;$
	\item ${}_{u|t}q_x={}_up_x~{}_{t}q_{x+u},u,t\ge0.$
\end{enumerate}
\end{corollary}

\begin{definition}[个体$(x)$的整数与分数余命及期望]
	类似处理新生儿的寿命一样, 可将个体$(x)$的余命$T(x)$分为整数部分和小数部分.
设
\begin{align*}
     & T(x)=K(x)+S(x),
\end{align*}
其中$K(x)$是$T(x)$的整数部分, $S(x)$是$T(x)$的小数部分. 记
\begin{align*}
     & \mathring{e}_x\stackrel{\text{def}}{=}E(T(x)), \   {e_x}\stackrel{\text{def}}{=}E(K(x)).
\end{align*}
则简单计算可知 \begin{align*}
    \mathring{e}_x =E(T(x))=\int_{0}^{\infty}{_tp_xdt}, ~
    {e_x}          =E(K(x))=\sum_{k=1}^{\infty}{_kp_x}.
\end{align*}
\end{definition}

\section{随机生存群}
\begin{definition}[模型描述]
	设0时刻系统中有$l_0$个新生儿, 他们的寿命独立同分布, 服从某分布, 生存函数为$s(t),t\ge 0.$  记

$\mathscr{L}(x)$ 为在$x$ 岁还活着的总人数;

$_t \mathscr{D}_x$为$[x,x+t]$内死去的总人数.

设系统中初始时刻的$l_0$个人的寿命分别为$X_1,X_2,...,X_n$, 则他们独立同分布, 且
$$P(X_i>t)=s(t), i=1,...,n.$$
显然有
\begin{align*}
     & \mathscr{L}(x)=\sum_{i=1}^{l_0}I_{\{X_{i}\geqslant x\}},\ {}_t\mathscr{D}_x=\sum_{i=1}^{l_0}I_{\{ x\leqslant X_i<x+t\}},
\end{align*}
其中
$
    I_{A}=\left\{\begin{array}{ll}1,&\omega\in A,\\0,&\omega\in A^c\end{array}\right.
$ 为示性函数.

令 \begin{align*}
    l_x     & \overset{def}{=}E(\mathscr L(x)), \text{ 它表示在}x\text{岁还活着的期望人数};         \\
    {}_td_x & \overset{def}{=}E({}_t\mathscr D_x ),\text{ 它表示在}[x,x+t)\text{内死去人数的期望}.
\end{align*}
\end{definition}

\begin{corollary} 如下结论成立:
	\begin{enumerate}
		\item $l_x=l_0s(x),\ {}_td_x=l_x-l_{x+t};$
		\item ${}_tp_x=\frac{l_{x+t}}{l_x};$
		\item ${}_tq_x=\frac{{}_td_x}{l_x};$
		\item $l_{x+t}=l_xe^{-\int_{x}^{x+t}\mu(s)ds};$
		\item $\frac{dl_x}{dx}=-l_x\mu(x);$
		\item $_nd_x=\int_{x}^{x+n}l_y\mu(y)dy.$
	\end{enumerate}
\end{corollary}

\begin{remark} 下面我们分析等式
	${}_td_x=\int_x^{x+t} l_yu(y)dy$的含义.
	
	等式左端${}_td_x$表示在$[x,x+t]$内死去的人数. 现分析右端. 注意到$$\mu(y)dy=-\dfrac{s'(y)}{s(y)}dy=-\dfrac{ds(y)}{s(y)}=\dfrac{s(y)-s(y+\Delta y)}{s(y)}.$$ 所以$\mu(y)dy$表示一个人在$y$岁还活着的条件下, 在$[y,y+dy]$内死去的概率, 于是$l_y\mu(y)dy$表示在$[y,y+dy]$内死去的人数. 对$y$积分可知, 等式右端的$\int_x^{x+t}l_y\mu(y)dy$表示在$[x,x+t]$内死去的人数. 所以右端等于左端.
\end{remark}

\section{生命表的元素}
\begin{proposition}
	在精算学的诸多记号中, 若左下标是1, 通常将其省略, 所以我们有
$${p_x}\triangleq{}_1p_x,~{q_x}\triangleq{}_1q_x,$$
$$L_x\triangleq{}_1L_x,~{}_{u|}q_x\triangleq{}_{u|1}q_x.$$
\end{proposition}

\begin{proposition}
	记$a\wedge b=\min\{a,b\},\ a\vee b=\max\{a,b\},\ EX=\int_0^{\infty}xf(x)dx,X\ge0.$ 计算可得
	\begin{equation*}
		E(X\wedge t) = \int_0^txf(x)dx+tP(X>t).
	\end{equation*}
\end{proposition}

\begin{definition}[条件数学期望]
	设$A$是一个随机事件, $X$为一个随机变量, 给定事件$A$的条件下, $X$的条件期望定义为
$$E(X|A)\overset{def}{=}\frac{E(XI_A)}{EI_A}=\frac{E(XI_A)}{P(A)}.$$
可以类似条件概率的定义理解条件期望的定义.
\end{definition}

\begin{definition}
	${}_tL_x:$ 所有人在$[x,x+t)$内活过的总时间, 记作$ {}_tL_x=l_xE(T(x)\wedge t).$
\end{definition}

\begin{corollary} 如下结论成立:
    \begin{align*}
        {}_tL_x & =l_xE(T(x)\wedge t)                 \\
                & =\int_0^t sl_{x+s}u(x+s)ds+tl_{x+t} \\
                & =\int_0^tl_{x+s}ds
    \end{align*}
\end{corollary}

\begin{remark}
	表达式$\int_0^tsl_{x+s}\mu(x+s)ds+tl_{x+t}$的含义如下:

	一方面, 在$x+s$岁活着的人有$l_{x+s}$个, 每个人在$[x+s,x+s+ds]$内死去的概率为$\mu(x+s)ds,$ 所以, 在$[x+s,x+s+ds]$内死去的人数为$l_{x+s}\mu(x+s)ds$, 他们每个人在$[x,x+t]$内活$s$年, 所以在$x+s$岁死去的人在$[x,x+t]$ 内活的总时间为$sl_{x+s}\mu(x+s)ds,$ 再对$s$在$(0,t)$求积分(求和)可知, $\int_0^tsl_{x+s}\mu(x+s)ds$表示在$[x,x+t]$内死去的人在这段时间内活过的总时间. 
	
	另一方面, 在$x+t$岁还活着的人有$l_{x+t}$个, 他们每个人在$[x,x+t)$内活了$t$岁, 故他们在$[x,x+t)$内总共活了$tl_{x+t}$岁.

	综合以上分析, $\int_0^tsl_{x+s}\mu(x+s)ds+tl_{x+t}$表示所有人在$[x,x+t]$内活过的总时间, 这是一个复杂的公式, 但它有一个简单的表达$\int_0^tl_{x+s}ds.$
\end{remark}

\begin{definition}
	$a(x):$ 一个$x$岁的人在1年内死去的条件下, 在$[x,x+1)$内活过的期望时间, 记作$a(x)=E(T(x)|T(x)\leq 1).$
\end{definition}

\begin{corollary}以下等式成立:
    \begin{enumerate}
		\item $a(x)=\dfrac {\int_0^1t{}_tp_xu_x(t)dt}{q_x};$
		\item $L_x=d_xa(x)+l_{x+1}.$
	\end{enumerate} 
\end{corollary}

\begin{remark}
	表达式$L_x=d_xa(x)+l_{x+1}$的含义如下:
   
	等式右端的$d_xa(x)$表示$[x,x+1)$内死去的人在$[x,x+1)$内活过的总时间. 右端的$l_{x+1}\times1$ 表示在$x+1$岁活着的$l_{x+1}$人在$[x,x+1)$内活过的总时间. 所以, 右端表示所有人在$[x,x+1)$活过的总时间, 正好等于左端的$L_x.$
   \end{remark}

\begin{proposition}
	\begin{enumerate}
		\item 中心死亡率: ${}_nm_x=\dfrac {{}_nq_x}{\int_0^n{}_tp_xdt}=\dfrac{{}_nd_x}{{}_nL_x};$
		\item $T_x\triangleq\int_0^{\infty}l_{x+s}ds={}_{\infty}L_x$, $T_x$表示所有人在$[x,\infty)$内活过的总时间; 
		\item $Y_x\triangleq\int_0^{\infty}T_{x+s}ds.$
	\end{enumerate}
\end{proposition}

\section{分数年龄上的死亡假设}
\begin{definition}[死亡力均匀分布假设(UDD假设)]
	若$x$为非负整数, $s(t)$是生存函数, 若$\forall t\in [0,1),$ 都有
\begin{align}\label{tula}
     & s(x+t)=(1-t)s(x)+ts(x+1).
\end{align}

称在$[x,x+1)$上, 死亡力均匀分布假设成立.
\end{definition}

\begin{corollary}
	设$[x,x+1)$上UDD假设成立, 则有以下结论:
	\begin{enumerate}
		\item $l_{x+t}=(1-t)l_{x}+tl_{x+1},t\in [0,1);$
		\item $_{t}d_{x}=td_{x},t\in [0,1);$
		\item $_{t}q_{x}=tq_{x},t\in [0,1);$
		\item $f_{T(x)}(t)=q_{x},t\in [0,1);$
		\item $\mu_{x}(t)=\frac{q_{x}}{1-tq_x}.$
	\end{enumerate}
\end{corollary}

\begin{proposition}
	在UDD假设之下, 我们有如下两个命题:
	\begin{enumerate}
		\item 已知在每一年龄年上UDD假设成立, 则$K(x)$与$S(x)$相互独立, 且$S(x)$ 服从$[0,1]$上的均匀分布;
		\item 在每一年龄年UDD假设成立时, 有
    $$\mathring{e}_{x}=e_{x}+\frac{1}{2},\   D(T(x))=D(K(x))+\frac{1}{12}.$$
	\end{enumerate}
\end{proposition}

\begin{definition}[常数死亡力假设]
	设$x$为整数, 若$\forall t\in [0,1)$有
\begin{align}\label{tulb}
    \ln s(x+t)=(1-t)\ln s(x)+t\ln s(x+1).
\end{align}
则称生存函数在年龄段$[x,x+1)$满足常数死亡力假设.
\end{definition}

\begin{corollary}
	设在年龄段$[x,x+1)$常数死亡力假设成立, 则对$t\in (0,1)$, 有
    \begin{enumerate}
		\item 期望生存人数满足$\ln l_{x+t}=(1-t)\ln l_{x}+t\ln l_{x+1};$
		\item 死亡力为常数, 即$\mu _{x}(t)=-\ln p_{x}\stackrel{\triangle}{=}\mu;$
		\item $l_{x+t}=l_{x}e^{-\mu t},\ _{t}q_{x}=1-p^{t}_{x},\ f_{T(x)}(t)=-p_{x}^{t}\ln p_{x}.$
	\end{enumerate}
\end{corollary}

\begin{example}
    设$S(x)=1-\frac{x}{12},\ 0\leq x\leq 12$, $l_{0}$个个体相互独立, 生存函数都是$S(x)$.

    (1) 求$(_{3}\mathscr D _{0},\ _{3}\mathscr D_{3},\ _{3}\mathscr D _{6},\ _{3}\mathscr D _{9})$的联合分布;

    (2) 求这四个随机变量的期望和方差;

    (3) 求它们两两之间的相关系数.
\end{example}

\begin{solution}
易知$l_{0}$个人的寿命$X_{1},X_{2},...,X_{l_{0}}\stackrel{\text{i.i.d.}}{\sim}U[0,12]$. 且随机变量满足
\begin{align*}
  {}_{3}\mathscr D _{0}=\sum^{l_{0}}_{k=1}I_{\{0\leq X_k \leq 3\}},~{}_{3}\mathscr D _{3}=\sum^{l_{0}}_{k=1}I_{\{3\leq X_k \leq 6\}},~{}_{3}\mathscr D _{6}=\sum^{l_{0}}_{k=1}I_{\{6\leq X_k \leq 9\}},~{}_{3}\mathscr D _{9}=\sum^{l_{0}}_{k=1}I_{\{9\leq X_k \leq 12\}}.
\end{align*}

(1) 令事件$A=\{_{3}\mathscr D _{0}=k_{1},\ _{3}\mathscr D _{3}=k_{2},\ _{3}\mathscr D _{6}=k_{3},\ _{3}\mathscr D _{9}=k_{4}\}$, 若事件$A$发生, 则在$l_{0}$个人中, 有$k_{1}$人
在$[0,3]$内死亡; 有$k_{2}$人在$[3,6]$内死亡; 有$k_{3}$人在$[6,9]$内死亡; 有$k_{4}$人在$[9,12]$内死亡, 其中$k_{1}+k_{2}+k_{3}+k_{4}=l_{0}.$ 从$k_{1}+k_{2}+k_{3}+k_{4}$个人中, 选出$k_{1}$个人在$[0,3]$内死亡, 有$C_{k_{1}+k_{2}+k_{3}+k_{4}}^{k_{1}}$种选法; 从$k_{2}+k_{3}+k_{4}$个人中, 选出$k_{2}$个人在$[3,6]$内死亡, 有$C_{k_{2}+k_{3}+k_{4}}^{k_{2}}$种选法; 从$k_{3}+k_{4}$个人中, 选出$k_{3}$个人在$[6,9]$内死亡, 有$C_{k_{3}+k_{4}}^{k_{3}}$种选法. 于是
$$
    P(A)=\frac{(k_{1}+k_{2}+k_{3}+k_{4})!}{k_{1}!k_{2}!k_{3}!k_{4}!}\cdot {}_{3}q_{0}^{k_{1}}\cdot {}_{3|3}q_{0}^{k_{2}}\cdot {}_{6|3}q_{0}^{k_{3}}\cdot {}_{9|3}q_{0}^{k_{4}}.
$$

(2) 对于一个二项分布$B(n, p)$, 其期望$E[X] = np$, 方差$\text{Var}(X) = np(1-p)$.

因此, 对于每个随机变量$_{3}\mathscr D_{k}$有

期望 $$E(_{3}\mathscr D_{k}) = \frac{l_0}{4};$$

方差 $$\text{Var}(_{3}\mathscr D_{k}) = l_0 \cdot \frac{1}{4} \cdot \left(1 - \frac{1}{4}\right) = \frac{3l_0}{16}.$$

(3) 以$_{3}\mathscr D _{0},\ _{3}\mathscr D _{3}$的相关系数为例:
\begin{equation*}
\begin{aligned}
    \text{Cov}(_{3}\mathscr D _{0},\ _{3}\mathscr D _{3}) & =E({}_{3}\mathscr D _{0}\ _{3}\mathscr D _{3})-E(_{3}\mathscr D _{0})E(_{3}\mathscr D _{3})                                \\
                                                          & =E(\sum ^{l_{0}}_{k=1}I_{\{0\leq X_{k}\leq 3\}}\cdot \sum ^{l_{0}}_{j=1}I_{\{3\leq X_{j}\leq 6\}})-(\frac{l_{0}}{4})^{2} \\
                                                          & =\sum^{l_{0}}_{k=1}\sum_{j\neq k}E(I_{\{0\leq X_{k}\leq 3\}}\cdot I_{\{3\leq X_{j}\leq 6\}})-(\frac{ l_{0}}{4})^{2}      \\
                                                          & =\frac{l_{0}(l_{0}-1)}{16}-\frac{l_{0}^{2}}{16}  =-\frac{l_{0}}{16}.
\end{aligned}
\end{equation*}
(对于$\sum^{l_{0}}_{k=1}\sum_{j\neq k}E(I_{\{0\leq x_{X}\leq 3\}}\cdot I_{\{3\leq X_{j}\leq 6\}}) = \frac{l_{0}(l_{0}-1)}{16}$ 的理解: 其中$\sum^{l_{0}}_{k=1}\sum_{j\neq k}1 = l_0(l_0-1)$, 而$p(I_{c\leq X_{i}\leq c+3}) = \frac14$, 故$I_{0\leq X_{k}\leq 3}$ 与$I_{3\leq X_{j}\leq 6}$同时取1的概率为$\frac{1}{16}$.)

求出协方差后即可求相关系数:
\begin{align*}
    \rho(_{3}\mathscr D _{0},\ _{3}\mathscr D _{3}) & =\frac{\text{Cov}(_{3}\mathscr D_{0},\ _{3}\mathscr D _{3})}{\sqrt{D(_{3}\mathscr D _{0})}\cdot \sqrt{D(_{3}\mathscr D _{3})}} \\
                                                    & =\frac{-\frac{l_{0}}{16}}{\sqrt{\frac{3}{16}l_{0}\cdot \frac{3}{16}l_{0}}}                                                     \\
                                                    & =-\frac{1}{3}.
\end{align*}
类似计算可知, 两两之间所有相关系数皆为$-\frac{1}{3}$.
\end{solution}

\chapter{人寿保险}
\section{人寿保险概述}
\begin{equation*}
    \text{人寿保险}
    \begin{cases}
         & \text{生存保险} \\
         & \text{生死合险} \\
         & \text{死亡保险}
		\begin{cases}
         & \text{定期死亡保险}  \\
         & \text{终身死亡保险}                   \\
         & \text{延期死亡保险}
		 \begin{cases}
         & \text{延期定期死亡保险} \\
         & \text{延期终身死亡保险}
    	 \end{cases}
    	\end{cases}
    \end{cases}
\end{equation*}

\section{生存保险}
\begin{definition}[支付现值]
	若被保险人在$n$年内死亡(即$T(x)<n$), 则不予任何支付;

若他在$n$年内未死(即$T(x)\geqslant n$), 则在$n$ 时刻支付他1元保险金.

若$T(x)<n$, 则$Z=0$; 若$T(x)\geqslant n$, 则$Z=1\cdot \nu^n=\nu^n$, 其中贴现因子$\nu=\frac{1}{i+1}.$

即$Z=\nu^nI_{\left\{ T\left( x \right) \geqslant n, \right\}}$.
\end{definition}

\begin{proposition}[精算现值与方差]
	\begin{enumerate}
		\item $E(Z)=\nu^n\cdot{}_np_x=A_{x:}\overset{1}{{}\annu{n}}={}_nE_x;$
		\item $E Z^2 =E\left( \left[ \nu^nI_{\left\{ T\left( x \right) \geqslant n \right\}} \right] ^2 \right) =E\left( \nu^{2n}I_{\left\{ T\left( x \right) \geqslant n \right\}} \right) =\nu^{2n}\cdot{}_np_x;$
		\item $DZ=EZ^2-\left( EZ \right) ^2=\nu^{2n}\cdot{}_np_x\cdot{}_nq_x.$
	\end{enumerate}
\end{proposition}

\begin{corollary}[精算现值的性质]
	$\forall 0\leqslant k\leqslant n$, 有
	\begin{enumerate}
		\item $_nE_x={}_kE_x\cdot {}_{n-k}E_{x+k}$;
		\item $(1+i)^k\cdot l_x\cdot{}_nE_x=l_{x+k}\cdot {}_{n-k}E_{x+k}$.
	\end{enumerate}
\end{corollary}

\begin{remark}
    等式$(1+i)^k\cdot l_x\cdot {}_nE_x=l_{x+k}\cdot {}_{n-k}E_{x+k}$ 的含义如下:

    在$0$时刻, $l_x$个人各自买了一份$n$年期的生存保险, 保费总额为$l_x\cdot {}_nE_x$, $k$年后, 这笔钱的积累值为$(1+i)^k\cdot l_x\cdot {}_nE_x$,若此时保险公司破产不干了, 他分给在$k$时刻还活着的$l_{x+k}$个人每人${}_{n-k}E_{x+k}$元, 这正好够每个人去重新买一份$n-k$年期的生存保险.
\end{remark}

\section{$n$年期(定期)死亡保险}

\subsection{死亡立即支付的$n$年期定期寿险}
\begin{definition}[支付现值]
	若$(x)$在$n$年内死亡(即$T(x)<n$), 则在$T(x)$时刻支付$1$元保险金;

若$(x)$在$n$年内未死(即$T(x)\geqslant n$), 则不予支付.

若$T(x)<n$, 则$Z=\nu^{T(x)}$; 若$T(x)\geqslant n$, 则$Z=0$.

所以$Z=\nu^{T(x)}I_{\{T(x)<n\}}.$
\end{definition}

\begin{proposition}[精算现值与方差]
	\begin{enumerate}
		\item $E(Z)=\int_0^{n}{\nu^{t}\cdot {}_tp_x\cdot \mu_{x}(t)dt}=\int_0^{n}{e^{-\delta t}\cdot {}_tp_x\cdot \mu_{x}(t)dt}=\overline{A}_{x:}^1{}_{{}\annu{n}};$
		\item $DZ={}^2\overline{A}_{x:\annu{n}}^1-(\overline{A}_{x:\annu{n}}^1)^2=\overline{A}_{x:\annu{n}}^1@2\delta-(\overline{A}_{x:\annu{n}}^1@\delta)^2.$
	\end{enumerate}
\end{proposition}

\begin{remark}
	记$^j\overline{A}_{x:}^1{}_{\annu{n}}=\int_0^{\infty}{e^{-j\delta t}\cdot {}_tp_x\cdot \mu_{x}(t)dt}$, $^j\overline{A}_{x:}^1{}_{\annu{n}}@\delta=\overline{A}_{x:}^1{}_{\annu{n}}@j\delta.$
\end{remark}

\begin{corollary}[精算现值的性质]
	$\forall 0\leqslant k\leqslant n$, 有
	\begin{enumerate}
		\item $\overline{A}_{x:}^1{}_{{}\annu{n}}=\overline{A}_{x:}^1{}_{\annu k}+{}_kE_x\cdot\overline{A}_{x+k:}^1{}\annu{n-k};$
		\item $l_x\cdot\overline{A}_{x:\annu{n}}^1=\int_0^{n}{\nu^{t}\cdot l_{x+t}\cdot \mu_{x}(t)dt}.$
	\end{enumerate}
\end{corollary}

\begin{remark}
	等式$l_x\cdot\overline{A}_{x:}^1{}_{\annu{n}}=\int_0^{n}{\nu^{t}\cdot l_{x+t}\cdot \mu_{x}(t)dt}$的含义如下::

    $\mu_x(t)dt$表示在$[x+t,x+t+dt]$内死去的概率, 所以$l_{x+t}\mu_x(t)dt$表示在$[x+t,x+t+dt]$内死去的人数, 在这期间内死去的人每人支付1元保险金, 共$l_{x+t}\mu_x(t)dt$ 元, 这些钱的现值为$\nu^t l_{x+t}\mu_x(t)dt$, 于是$\int_0^n{\nu^t l_{x+t}\mu_x(t)dt}$ 表示在$[x,x+n]$内死去的人领取的保险的总现值, 这些钱应等于初始时刻的$l_x$个人的保费总额$l_x\overline{A}_{x:}^1{}_{\annu{n}}$.
\end{remark}

\begin{example}
    假设死亡力$\mu(t) \equiv \mu $, 利息力为$\delta$, 个体$(x)$投了一个$n$年期寿险, 计算精算现值及支付现值的方差.
\end{example}

\begin{solution}
	由于死亡力是一个常数, 所以
\begin{align*}
    \overline{A}_{x:}^1{}_{\annu{n}} & = \int_0^n{e^{-\delta t}{}_tp_x\mu_x(t)dt}= \int_0^n{e^{-\delta t}\cdot e^{-\int_{x}^{x+t} \mu ds}\mu dt} \\
	     & = \int_0^n{e^{-\delta t}\cdot e^{-\mu t}\mu dt} = \frac{\mu}{\mu + \delta} \left(1 - e^{-(\mu + \delta)n} \right).
\end{align*}
进而有
\begin{align*}
    ^2\overline{A}_{x:}^1{}_{\annu{n}}= \frac{\mu}{\mu + 2\delta} \left(1 - e^{-(\mu + 2\delta)n} \right).
\end{align*}
于是
\begin{align*}
    DZ& = {}^2\overline{A}_{x:\annu{n}}^1 - (\overline{A}_{x:\annu{n}}^1)^2\\
     &= \frac{\mu}{\mu + 2\delta} \left(1 - e^{-(\mu + 2\delta)n} \right) - \left(\frac{\mu}{\mu + \delta} \left(1 - e^{-(\mu + \delta)n} \right)\right)^2.
\end{align*}
\end{solution}

\subsection{死亡年末支付的$n$年期定期寿险}
\begin{definition}[支付现值]
	若个体$(x)$在$n$年内死亡, 则在其死亡年末支付$1$元; 

若个体$(x)$在$n$年内未死, 则不予支付.

$Z=\nu^{K(x)+1}I_{\{T(x)<n\}}$.
\end{definition}

\begin{proposition}[精算现值与方差]
	\begin{enumerate}
		\item $E(Z) = \sum_{k=0}^{n-1}\nu^{k+1}{}_{k|}q_x = A_{x:}^1{}_{\annu{n}};$
		\item $DZ = {}^2A_{x:}^1{}_{\annu{n}} - (A_{x:}^1{}_{\annu{n}})^2$, 其中${}^2A_{x:}^1{}_{\annu{n}} = \sum_{k=0}^{n-1}\nu^{2(k+1)}{}_{k|}q_x.$
	\end{enumerate}
\end{proposition}

\begin{corollary}[精算现值的性质]
	$\forall~0\leq m\leq n,$ 有$A_{x:}^1{}_{\annu{n}}=A_{x:}^1{}_{\annu m}+{}_mE_x\cdot A_{x+m:}^1{}_{\annu {n-m}}$.

	注意到当$m=1$时, $A_{x:}^1{}_{\annu{1}} = \nu q_x,~{}_1E_x = \nu p_x$, 立即可得
	\begin{enumerate}
		\item $A_{x:}^1{}_{\annu{n}}=\nu q_x+\nu p_x\cdot A_{x:}^1{}_{\annu{n-1}}$;
		\item $(1+i)l_x A_{x:}^1{}_{\annu{n}} = d_x + l_{x+1} A_{x+1:}^1{}_{\annu{n-1}}.$
	\end{enumerate}
\end{corollary}

\begin{remark}
	等式$(1+i)l_x A_{x:}^1{}_{\annu{n}} = d_x + l_{x+1} A_{x+1:}^1{}_{\annu{n-1}}$ 的含义如下:
  
  此式的左端表示在0时刻, $l_x$个人各自买了一份$n$年期的死亡保险, 保费总额为$l_x A_{x:\annu{n}}^1$. 一年后, 这笔钱的积累值为$(1+i)l_x A_{x:}^1{}_{\annu{n}}$. 右端表示, 在$[0,1]$之间有$d_x$个人死去, 保险公司需给他们每人$1$元, 共$d_x$元. 若此时保险公司破产不干了, 他分给在1时刻还活着的$l_{x+1}$个人每人$A_{x+1:}^1{}_{\annu{n-1}}$元, 这正好够每个人去重新买一份$n-1$年期的死亡保险. 所以保险公司一年后的支出总额为$d_x + l_{x+1} A_{x+1:}^1{}_{\annu{n-1}},$ 正好等于保险公司收到总保费在一年后的累计值. 所以左端等于右端.
  \end{remark}

\begin{example}
    设一个$20$岁的人买了一份$10$年期的死亡保险, 设其余命$T(20)$服从$[0,80]$上的均匀分布, $i = 0.05$, 保险金死亡年末支付$10$万元.
\end{example}
\begin{solution}
	支付现值
    $$
        Z = \nu^{K(20)+1}I_{T(20)<10}\cdot 100000,
    $$
    精算现值
    \begin{align*}
        E(Z) & = 100000\cdot A_{20:}^1{}\annu{10}                         \\
             & = 100000\cdot \sum_{k=0}^{9}{\nu^{k+1}\cdot {}_{k|}q_{20}} \\
             & \approx 9652.1687.
    \end{align*}
\end{solution}

\begin{proposition}[死亡年末支付与死亡立即支付的关系]
	设死亡力均匀分布假设成立, 则$\overline{A}_{x:}^1{}_{\annu{n}}= \frac{i}{\delta}A_{x:}^1{}_{\annu{n}}$.
\end{proposition}

\section{终身死亡保险}
\subsection{死亡后立即支付的终身死亡保险}
\begin{definition}[支付现值]
	在个体$(x)$死亡时刻立刻支付$1$元保险金, $Z = \nu^{T(x)}.$
\end{definition}

\begin{proposition}[精算现值与方差]
	\begin{enumerate}
		\item $E(Z) = \int_0^\infty \nu^t{}_tp_x\mu_x(t)\mathrm{d}t = \overline{A}_x;$
		\item $DZ = {}^2\overline{A}_x - (\overline{A}_x)^2$, 其中${}^2\overline{A}_x = \int_0^\infty \nu^{2t}{}_tp_x\mu_x(t)\mathrm{d}t$.
	\end{enumerate}
\end{proposition}

\begin{corollary}[精算现值的性质]
	\begin{enumerate}
		\item 对$n\ge1,$ 有$\overline{A}_x = \overline{A}_{x:}^1{}_{\annu{n}} + {}_nE_{x}\cdot \overline{A}_{x+n}$;
		\item $\frac{d\overline{A}_x}{dx} = \delta \overline{A}_x + \mu(x)(\overline{A}_x - 1).$
	\end{enumerate}
\end{corollary}

\begin{example}
    设死亡力为$\mu$, 利息力为$\delta$,个体$(x)$投了一份死亡立即支付的终身寿险, 求$\overline{A}_{x}, {}^2\overline{A}_{x} $和给付现值$Z$的方差$D(Z)$.
\end{example}
    \begin{solution}
    \begin{align*}
        \overline{A}_{x}     & =\int_0^{\infty}e^{-\delta t}\cdot {}_tp_x\cdot \mu_x(t)dt \\
        & =\int_0^{\infty}e^{-\delta t}\cdot e^{-\int_x^{x+t}\mu(s)ds}\cdot \mu_x(t)dt \\
        & =\int_0^{\infty}e^{-\delta t}\cdot e^{-\int_x^{x+t}uds}\cdot \mu dt    \\
        & =\int_0^{\infty}e^{-\delta t}\cdot e^{-ut}\cdot \mu dt \\
        & =\frac{\mu}{\delta +\mu}, \\
        {}^2\overline{A}_{x} 
		& =\frac{\mu}{2\delta +\mu}, \\
        DZ & ={}^2\overline{A}_{x}-(\overline{A}_{x})^2=\frac{\mu}{2\delta +\mu}-(\frac{\mu}{\delta +\mu})^2.
    \end{align*}
\end{solution}

\subsection{死亡年末支付的终身寿险}
\begin{definition}[支付现值]
	在个体$(x)$死亡的年末, 保险人支付$1$元保险金, $Z=\nu^{K(x)+1}$.
\end{definition}

\begin{proposition}[精算现值与方差]
	\begin{enumerate}
		\item $E(Z) = \sum_{k=0}^{\infty }{\nu^{k+1}{}_{k|}q_x} = A_x;$
		\item $DZ = {}^2\overline{A}_x - (\overline{A}_x)^2$, 其中${}^2\overline{A}_x = \int_0^\infty \nu^{2t}{}_tp_x\mu_x(t)\mathrm{d}t$.
	\end{enumerate}
\end{proposition}

\begin{corollary}[精算现值的性质]
	\begin{enumerate}
		\item ${A}_x = {A}_{x:}^1{}_{\annu{n}} + {}_nE_{x}\cdot {A}_{x+n}$;
		\item ${A}_x = \nu q_x + \nu p_x\cdot {A}_{x+1};$
		\item $(1+i){A}_x = q_x + p_x\cdot {A}_{x+1};$
		\item $(1+i)l_x{A}_x = d_x + l_{x+1}\cdot {A}_{x+1};$
		\item $l_x{A}_x = \sum_{k=0}^{\infty }{\nu^{k+1}\cdot d_{x+k}}.$
	\end{enumerate}
\end{corollary}

\begin{proposition}
	在UDD假设之下, 有$\overline A_x=\dfrac{i}{\delta}A_x$.
  \end{proposition}

\section{生死合险(两全保险)}

\section{延期终身死亡保险}
\subsection{死亡立即支付的延期终身死亡保险}
% \begin{definition}
% 	若个体$(x)$在$m$年内死亡(即$T(x)<m$), 则不予任何支付;

% 若个体$(x)$在$m$年后还活着(即$T(x) \ge m$), 则在死亡时立即支付$1$元保险金.

% $Z=\nu^{T(x)}\cdot I_{\{T(x)\ge m\}}$.
% \end{definition}















\section{将每年分为$m$个区间, 在死亡区间末支付$1$元的终身死亡保险}

\section{变额人寿保险}

\section{小结}
\subsection{寿险小结}
\begin{center}
    \begin{tabular}{c|c|c|c}
        \hline
        & 期初寿险 & 期末寿险 & 连续寿险 \\
        \hline
        终身 & $\ddot a_{\annu{K(x)+1}} = \sum_{j=0}^{K(x)}{\nu^j}$ & $a_{\annu{K(x)}} = \sum_{j=1}^{K(x)}{\nu^j}$ & $\overline{a}_{\annu{T(x)}}=\int_0^{T(x)} \nu^tdt$ \\
        \hline
        $n$年期 & $\ddot a_{\annu{(K(x)+1)\land n}} = \sum_{j=0}^{K(x)\land (n-1)}{\nu^j}$ & $a_{\annu{K(x)\land n}} = \sum_{j=1}^{K(x)\land n}{\nu^j}$ & $  \overline{a}_{\annu{T(x)\land n}}=\int_0^{T(x)\land n} \nu^tdt$ \\
        \hline
        延期$n$年 & $\ddot a_{\annu{(K(x)+1)}} - \ddot a_{\annu{(K(x)+1)\land n}}$ & $a_{\annu{K(x)}} - a_{\annu{K(x)\land n}}$ & $\overline{a}_{\annu{T(x)}} - \overline{a}_{\annu{T(x)\land n}}$  \\
        \hline
        $n$年期确定性 & $\ddot a_{\annu{(K(x)+1)\lor n}}$ & $a_{\annu{K(x)\lor n}}$ & $\overline{a}_{\annu{T(x)\lor n}}$  \\  
        \hline
    \end{tabular}
\end{center}



\chapter{生存年金}
在本章最后一章节给出了各种生存年金的定义公式及精算现值, 故在前面几个章节仅给出一些命题与性质. 
\section{期初生存年金}
\subsection{终身期初生存年金}
\begin{corollary}
	因为$Z=\ddot a_{\annu {K(x)+1}}=\frac{1-\nu^{K(x)+1}}{d}$, 所以$\ddot a_x=EZ=E(\ddot a_{\annu {K(x)+1}})=E(\frac{1-\nu^{K(x)+1}}{d})$. 故$\ddot a_{x}=\frac{1-E(\nu^{K(x)+1})}{d}=\frac{1-A_x}{d}$.
于是如下等式成立,
\begin{align*}
    A_x+d \ddot a_x=1.
\end{align*}
\end{corollary}


\section{期末生存年金}

\section{每年分成$m$个区间的生存年金}

\section{连续生存年金}


\section{小结}
\subsection{生存年金小结}
\begin{center}
    \begin{tabular}{c|c|c|c}
        \hline
        & 期初生存年金 & 期末生存年金 & 连续生存年金 \\
        \hline
        终身 & $\ddot a_{\annu{K(x)+1}} = \sum_{j=0}^{K(x)}{\nu^j}$ & $a_{\annu{K(x)}} = \sum_{j=1}^{K(x)}{\nu^j}$ & $\overline{a}_{\annu{T(x)}}=\int_0^{T(x)} \nu^tdt$ \\
        \hline
        $n$年期 & $\ddot a_{\annu{(K(x)+1)\land n}} = \sum_{j=0}^{K(x)\land (n-1)}{\nu^j}$ & $a_{\annu{K(x)\land n}} = \sum_{j=1}^{K(x)\land n}{\nu^j}$ & $  \overline{a}_{\annu{T(x)\land n}}=\int_0^{T(x)\land n} \nu^tdt$ \\
        \hline
        延期$n$年 & $\ddot a_{\annu{(K(x)+1)}} - \ddot a_{\annu{(K(x)+1)\land n}}$ & $a_{\annu{K(x)}} - a_{\annu{K(x)\land n}}$ & $\overline{a}_{\annu{T(x)}} - \overline{a}_{\annu{T(x)\land n}}$  \\
        \hline
        $n$年期确定性 & $\ddot a_{\annu{(K(x)+1)\lor n}}$ & $a_{\annu{K(x)\lor n}}$ & $\overline{a}_{\annu{T(x)\lor n}}$  \\  
        \hline
    \end{tabular}
\end{center}

\subsection{生存年金精算现值小结}
\begin{center}
    \begin{tabular}{ c|c|c|c }
        \hline
        $ $ & 期初生存年金& 期末生存年金      & 连续生存年金   \\
         \hline
        终身& $\ddot a_x=\sum_{j=0}^{\infty}{\nu^j_jp_x}$ & $a_{x}=\ddot a_{x}-1=\sum^{\infty}_{j=1}\nu^{j}{}_jp_x$      & $\overline{a}_{x}=\int_0^{\infty} \nu^t{}_{t}p_xdt$           \\
        \hline
          $n$年期& $\ddot a_{x:}{}{\annu n}=\sum_{j=0}^{n-1}{\nu^j}{}_j p_x$ & $ \begin{aligned}
          a_{x:}{}\annu n&=\sum^{n}_{j=1}\nu^{j}{}_jp_x\\
          &=\ddot a_{x:}{}\annu n-1+\nu^{n}{}_{n}p_x
          \end{aligned}$ & $  \overline{a}_{x:}{}\annu{n}=\int_0^{n} \nu^t{}_{t}p_xdt$  \\
           \hline
          延期$n$年& $\begin{aligned}
          _{n|}\ddot a_{x}&=\ddot a_{x}-\ddot a_{x:}{}\annu n\\
          &={}_{n}E_{x}\ddot a_{x+n}\\
          &=\sum^{\infty}_{j=n}\nu^{j}{}_{j}p_{x}\\
          \end{aligned}$ & $\begin{aligned}
 _{n|}a_{x}& = a_{x}-a_{x:}{}\annu n \\
& = {}_{n}E_{x}a_{x+n}\\
&=\sum_{j=n+1}^{\infty}\nu^{j}{}_{j}p_{x}
\end{aligned}$ & $\begin{aligned}
          {}_{n|}\overline{a}_{x}&=\int_n^{\infty} \nu^t{}_{t}p_xdt\\
          &=\overline{a}_{x}-\overline{a}_{x:}{}\annu{n}\\
          &={}_nE_x\overline{a}_{x+n}
          \end{aligned}$ \\
           \hline
          $n$年期确定型& $\ddot a_{\overline{x:\annu n}}={}_{n|}\ddot a_{x}+\ddot a_{\annu n}$ & $a_{\overline{x:\annu{n}}}=a_{\annu n} + {}_{n|}a_{x}$ & $\overline a_{\overline{x:\annu{n}}}={}_{n|}\overline{a}_{x}+\overline a_{\annu {n}}$ \\
        \hline
    \end{tabular}
\end{center}


\chapter{净保费理论}
\section{平衡准则}

\section{趸交净保费}

\section{完全连续险种的年均衡净保费}

\section{完全离散险种的年均衡净保费}


\chapter{净准备金理论}
\section{确定净准备金的准则}

\section{完全连续险种在平衡准则下的净准备金}

\section{完全离散险种的净准备金}







\end{document}